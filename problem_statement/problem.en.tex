\problemname{Cities of Finland}

\illustration{.2}{finland}{Photo by \href{https://www.mapsopensource.com}{mapsopensource.com}}
You're a cartographer tasked with designing a new political map of Finland. In order 
to make your map easy to read, you've decided to group cities by their population.
Larger cities will have larger fonts and smaller cities will have smaller fonts.

In order for this scheme to work, you need a metric to divide the cities into population
ranges that reflect their relative size. You want your ranges to reflect how big cities
are relative to one another, so similarly sized cities should be members of the
same range. In order to get the best ranges possible, you want to minimize the total
standard deviation over all of the ranges.

For a given range of cities \(k = k_1, k_2, ..., k_m\) (where \(k_i\) is the population
of the ith city) the average population is \[\bar{k} = \frac{\sum_{i = 1}^m k_i}{m}.\]

The standard deviation for that range can be calculated as follows
\[\sqrt{\frac{\sum_{i = 1}^m (k_i - \bar{k})^2}{m}}.\]
\section*{Input}
The first line contains two space separated integers \(c\) and \(n\) (\(1 \leq
c \leq 100\) and \(c \leq n \leq 500\)) where \(c\) is the number of ranges to divide
the \(n\) cities. The following \(n\) lines contain city descriptions with a city name
and population. All city names are unique. All populations are in the range \(1 \leq p
\leq 10^6\)

In the sample input, we want to divide 8 towns into three ranges. The best way to
do this groups the four small towns (Lappeenranta, Rovaniemi, Pori, and Joensu) and
the three medium sized towns (Tampere, Turku, and Oulu) leaving Helsinki in a range
by itself.

\section*{Output}
Print, in ascending order, the population ranges with the minimum total standard deviation
(there should be \(c\) lines of output). 
There will be one unique solution. The first range starts at \(0\) and goes to
the maximum population in that range. Every other range starts one past the last
range's maximum.
